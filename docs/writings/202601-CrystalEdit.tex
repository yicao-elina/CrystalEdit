% science_template.tex
% See accompanying readme.txt for copyright statement, change log etc.

% Any modification of this template, including writing a paper using it,
% MUST rename the file i.e. use a different file name.

%%%%%%%%%%%%%%%% START OF PREAMBLE %%%%%%%%%%%%%%%

% Basic setup. Authors shouldn't need to adjust these commands.
% It's annoying, but please do NOT strip these into a separate file.
% They need to be included in this .tex for our production software to work.

% Use the basic LaTeX article class, 12pt text
\documentclass[12pt]{article}

% Science uses Times font. If you don't have this installed (most LaTeX installations will be
% fine) or prefer the old Computer Modern fonts, comment out the following line
\usepackage{newtxtext,newtxmath}
% Depending on your LaTeX fonts installation, you might get better results with one or both of these:
% \usepackage{mathptmx}
%\usepackage{txfonts}

% Allow external graphics files
\usepackage{graphicx}

% Use US letter sized paper with 1 inch margins
\usepackage[letterpaper,margin=1in]{geometry}

% Yi Cao's additional 
\usepackage[version=4]{mhchem}
\usepackage{subcaption}
\usepackage{booktabs} % For better table formatting
% Double line spacing, including in captions
\linespread{1.5} % For some reason double spacing is 1.5, not 2.0!

% One space after each sentence
\frenchspacing

% Abstract formatting and spacing - no heading
\renewenvironment{abstract}
	{\quotation}
	{\endquotation}

% No date in the title section
% \date{}

% Reference section heading
\renewcommand\refname{References and Notes}

% Figure and Table labels in bold
\makeatletter
\renewcommand{\fnum@figure}{\textbf{Figure \thefigure}}
\renewcommand{\fnum@table}{\textbf{Table \thetable}}
\makeatother

% Call the accompanying scicite.sty package.
% This formats citation numbers in Science style.
\usepackage{scicite}

% Provides the \url command, and fixes a crash if URLs or DOIs contain underscores
\usepackage{url}

%%%%%%%%%%%% CUSTOM COMMANDS AND PACKAGES %%%%%%%%%%%%

% Authors can define simple custom commands e.g. as shortcuts to save on typing
% Use \newcommand (not \def) to avoid overwriting existing commands.
% Keep them as simple as possible and note the warning in the text below.
% Example:
\newcommand{\pcc}{\,cm$^{-3}$} % per cm-cubed

% Please DO NOT import additional external packages or .sty files.
% Those are unlikely to work with our conversion software and will cause problems later.
% Don't add any more \usepackage{} commands.


%%%%%%%%%%%%%%%% TITLE AND AUTHORS %%%%%%%%%%%%%%%%

% Title of the paper.
% Keep it short and understandable by any reader of Science.
% Avoid acronyms or jargon. Use sentence case.
\def\scititle{
	Project Update: CrystalEdit - Exploring LLM Capabilities and Limitations for Physics-Aware Crystal Manipulation
}
% Store the title in a variable for reuse in the supplement (otherwise \maketitle deletes it)
\title{\bfseries \boldmath \scititle}

% Author and institution list.
% Institution numbers etc. should be hard-coded, do *not* use the \footnote command.
\author{
	% You can write out first names or use initials - either way is acceptable, but be consistent
	Yi Cao$^{1\dagger}$,
	Paulette Clancy$^{1\ast\dagger}$\and
	% Additional lines of authors should be inserted using the \and command (not \\
	% Institution list, in a slightly smaller font
	\small$^{1}$Chemical and Biomolecular Engineering, Johns Hopkins University, Baltimore & 21218, USA.\and
	% \small$^{2}$Another Department, Different Institution, City & Postal Code, Country.\and
	% % Identify at least one corresponding author, with contact email address
	\small$^{\ast}$Corresponding author. Email: pclancy3@jhu.edu\and
	% % Joint contributions can be indicated like this
	% \small$^{\dagger}$These authors contributed equally to this work.
}

%%%%%%%%%%%%%%%%% END OF PREAMBLE %%%%%%%%%%%%%%%%


%%%%%%%%%%%%%%%% START OF MAIN TEXT %%%%%%%%%%%%%%%
\begin{document} 

% Insert the title and author list
\maketitle

% The first paragraph of any Science paper does NOT have a heading
% Nor is it indented
\noindent


\section{Executive Summary: The CrystalEdit Vision}
While generative AI has shown promise in \textit{de novo} crystal generation, the "Editing" problem—performing \textit{targeted, physics-constrained} modifications—remains challenging. Just as experimental scientists often optimize existing materials through doping or defect engineering rather than discovering entirely new ones, AI agents should be capable of performing similar targeted manipulations.

\textbf{CrystalEdit} aims to develop a benchmark and toolset that explores how LLM agents can move from \textit{understanding} how ML models interpret materials (our XAI work) to \textit{actively manipulating} those materials. The goal is to create an \textbf{"Active Intelligence"} component that bridges the gap between static datasets and the high-quality, non-equilibrium configurations needed for robust Machine Learning Force Fields (MLFFs).

Potential capabilities we are exploring:
\begin{itemize}
    \item \textbf{Addressing the "Data Gap":} Investigating whether LLMs can help generate high-quality, non-equilibrium synthetic data for MLFF training, potentially reducing the need for costly AIMD runs.
    \item \textbf{Workflow Automation:} Testing whether complex structure-building tasks (e.g., grain boundaries, twin defects) can be simplified through natural language interfaces.
    \item \textbf{Active Learning Integration:} Exploring connections with our XAI framework to identify and generate structures that maximize model learning.
\end{itemize}

\section{Strategic Alignment with Current Research}
This project builds upon our previous work while exploring new directions:

\begin{table}[h]
\centering
\begin{tabular}{@{}lp{10cm}@{}}
\toprule
\textbf{Previous Work} & \textbf{Connection to CrystalEdit} \\
\midrule
\textbf{NeurIPS (AI4Mat)} & We identified that MLFFs require diverse, non-equilibrium states. CrystalEdit explores whether LLM-assisted generation can systematically create such states (strains, defects). \\
\textbf{XAI (AAAI 2026)} & Our XAI framework reveals \textit{what} the model learns; CrystalEdit investigates whether we can \textit{test} those insights by specifically modifying identified features. \\
\textbf{General: Material Screening} & By exploring automated "Crystal Editing" processes, we aim to move beyond screening known databases toward systematically modified structures with controlled defects. \\
\bottomrule
\end{tabular}
\caption{Strategic alignment of CrystalEdit with ongoing research goals.}
\label{tab:alignment}
\end{table}

\section{Implementation Progress and Current Limitations}

\subsection{Stage 1: 0D Point Defects (Most Robust Performance)}
We have implemented automated insertion of interstitials and substitutions in the \ce{Sb2Te3} system, which shows the highest success rate among all defect types.
\begin{itemize}
    \item \textbf{Approach:} The agent identifies octahedral/tetrahedral voids in the van der Waals gap, with validation through ASE/Pymatgen checks.
    \item \textbf{Success Rate:} Point defect generation is relatively robust, with most failures occurring when the LLM misidentifies chemically inequivalent sites.
    \item \textbf{Limitation:} The system cannot yet automatically recognize all crystallographic variations of defect sites. For example, in structures with multiple inequivalent positions, significant prompt engineering is required to guide the LLM toward the correct site selection.
\end{itemize}

\subsection{Stage 2: 2D Surface \& Planar Engineering (Moderate Robustness)}
We have implemented basic capabilities for surface and planar defect generation, though with notable limitations.
\begin{itemize}
    \item \textbf{Slab Generation:} Controlled termination (Te vs. Sb) can be achieved with careful prompting (Figure \ref{fig:defects}A, B).
    \item \textbf{Stacking Faults:} Implementation of Shockley partial dislocations for metastable phase modeling (Figure \ref{fig:defects}C).
    \item \textbf{Challenges:} Instruction following becomes less robust for 2D defects. The LLM occasionally struggles with relative spatial positioning, particularly when multiple layers need coordinated manipulation. Success often requires iterative prompt refinement and incorporation of domain-specific physical knowledge into the context.
\end{itemize}

\begin{figure}[ht]
    \centering
    \begin{subfigure}{0.45\textwidth}
        \includegraphics[width=\linewidth]{figures/Fig_Slab_Te.pdf}
        \caption{Te-Terminated Slab}
    \end{subfigure}
    \hfill
    \begin{subfigure}{0.45\textwidth}
        \includegraphics[width=\linewidth]{figures/Fig_Slab_Sb.pdf}
        \caption{Sb-Terminated Slab}
    \end{subfigure}
    \hfill
    \begin{subfigure}{0.8\textwidth}
        \includegraphics[width=\linewidth]{figures/Fig_Fault.pdf}
        \caption{Stacking Fault}
    \end{subfigure}
    \caption{Planar defects generated by CrystalEdit. (A) Thermodynamically stable Te-terminated surface. (B) Metastable Sb-terminated surface. (C) Stacking fault modifying the quintuple layer sequence. Success rates vary with defect complexity.}
    \label{fig:defects}
\end{figure}

\subsection{Stage 3: Complex 3D Microstructures (Work in Progress)}
We have begun exploring Coincidence Site Lattice (CSL) theory implementation for high-order defects, with initial testing on BCC Ti twin boundaries.

\textbf{Collaborative Case Study:} This work originated from a collaboration request by Xiaofun Feng (EPFL), who needed twin plane structures for mechanical property studies. This provided an excellent test case for the system's capabilities and limitations.

\begin{itemize}
    \item \textbf{Twin Boundaries:} Attempted construction of coherent $\Sigma 3$ and $\Sigma 7$ boundaries with hierarchical \{233\} and \{322\} twin systems (Figure \ref{fig:twin_render}).
    \item \textbf{Human-in-the-Loop Requirement:} Generating physically correct twin structures required substantial back-and-forth interaction. Key success factors included:
    \begin{itemize}
        \item Providing well-documented literature references with clear methodological descriptions
        \item Iterative prompt engineering to incorporate crystallographic constraints
        \item Manual validation at each construction step
        \item Explicit injection of domain knowledge (e.g., twin plane orientations, CSL relationships)
    \end{itemize}
    \item \textbf{Script-Based Enhancement:} We found that leveraging Model Context Protocol (MCP) with script-based generation significantly improved robustness compared to pure natural language approaches. Pre-validated crystallographic operations reduce the LLM's burden of spatial reasoning.
    \item \textbf{Current Limitation:} The system cannot independently handle complex 3D defect variations. Success heavily depends on the quality of contextual information provided and the clarity of methodological documentation in reference materials.
\end{itemize}

\begin{figure}[ht]
    \centering
    \includegraphics[width=0.8\linewidth]{figures/twin_render_233_322.png}
    \caption{POV-Ray rendering of a hierarchical twin structure in BCC Ti, generated through iterative human-in-the-loop interaction. The image shows the primary \{233\} twin boundary plane (grey) and distinct atomic arrangements of twin domains (blue vs red). This structure was developed in collaboration with X. Feng (EPFL) and required multiple refinement cycles with literature-informed prompting.}
    \label{fig:twin_render}
\end{figure}

\section{Success Rate Analysis and Robustness Patterns}

Our preliminary assessment reveals a clear hierarchy in task difficulty:

\begin{table}[h]
\centering
\begin{tabular}{@{}lp{8cm}@{}}
\toprule
\textbf{Defect Type} & \textbf{Robustness Assessment} \\
\midrule
\textbf{0D (Point Defects)} & High robustness ($\sim$80-90\% success with minimal prompting); main failures from site identification \\
\textbf{2D (Surfaces/Faults)} & Moderate robustness ($\sim$50-70\% success); requires careful prompt engineering for spatial consistency \\
\textbf{3D (Grain Boundaries)} & Low robustness ($\sim$30-40\% success); heavily dependent on human guidance and reference quality \\
\bottomrule
\end{tabular}
\caption{Preliminary success rates for different defect generation tasks. Rates based on first-attempt success without iterative refinement.}
\label{tab:success_rates}
\end{table}

\textbf{Key Findings:}
\begin{itemize}
    \item \textbf{Spatial Reasoning Limitation:} LLMs struggle with relative spatial positioning in 3D, often confusing coordinate transformations or crystallographic relationships.
    \item \textbf{Context Dependence:} Performance dramatically improves when provided with well-documented methodological references, suggesting the need for curated knowledge bases.
    \item \textbf{Hybrid Approach Advantage:} Combining LLM reasoning with validated script libraries (via MCP) shows promise for improving reliability.
\end{itemize}

\begin{figure}[ht]
    \centering
    \includegraphics[width=0.8\linewidth]{figures/Fig2_Benchmark.pdf}
    \caption{Preliminary benchmarking of structural fidelity. The plot compares a composite 'Physics Score' (bond valence and steric validity) against geometric complexity. While point defects maintain high fidelity, complex operations show increased variability, highlighting the need for improved validation frameworks. Error bars represent variation across multiple generation attempts.}
    \label{fig:benchmark}
\end{figure}

\section{Lessons Learned and Future Directions}

\subsection{Current System Limitations}
\begin{enumerate}
    \item \textbf{Incomplete Defect Space Coverage:} The system cannot automatically enumerate all crystallographically distinct defect configurations.
    \item \textbf{Instruction Following Variability:} Performance degrades with increasing structural complexity, requiring extensive prompt engineering.
    \item \textbf{Domain Knowledge Injection:} Physical constraints must often be explicitly encoded rather than inferred by the LLM.
    \item \textbf{Validation Burden:} Significant human oversight remains necessary, particularly for novel or complex structures.
\end{enumerate}

\subsection{Promising Directions}
\begin{enumerate}
    \item \textbf{Hybrid Architectures:} Combining LLM reasoning with validated crystallographic operation libraries shows improved reliability.
    \item \textbf{Knowledge Base Development:} Curating high-quality reference materials with clear methodological descriptions enhances context quality.
    \item \textbf{Active Validation Loops:} Integrating real-time physical validation (symmetry checks, energy screening) as guardrails rather than post-hoc verification.
    \item \textbf{Task Stratification:} Focusing LLM capabilities on appropriate complexity levels while using traditional methods for highly complex operations.
\end{enumerate}

\section{Realistic Impact Assessment}

This month's work has established CrystalEdit as a \textit{proof-of-concept} for LLM-assisted crystal engineering, with clear evidence of both capabilities and limitations:

\textbf{Near-Term Value:}
\begin{itemize}
    \item Accelerating simple, repetitive defect generation tasks (particularly point defects)
    \item Providing natural language interfaces for basic structure manipulation
    \item Serving as an educational tool for understanding structure-property relationships
\end{itemize}

\textbf{Research Questions for Future Work:}
\begin{itemize}
    \item Can hybrid LLM-script approaches achieve reliability comparable to expert-written code?
    \item What level of domain knowledge injection is optimal for balancing flexibility and correctness?
    \item How can we systematically quantify the "physics understanding" of LLM-generated structures?
    \item Can active learning frameworks effectively guide LLMs toward generating scientifically valuable structures?
\end{itemize}

\section{Conclusion}
CrystalEdit represents an early-stage exploration of LLM-assisted crystal structure manipulation. While showing promise for simple tasks, the system's current limitations—particularly in spatial reasoning and instruction following for complex defects—highlight the need for careful integration of traditional computational methods with emerging AI capabilities. The collaborative case study with EPFL demonstrated both the potential and the practical challenges of deploying such systems in real research scenarios.

Moving forward, we will focus on developing hybrid approaches that leverage LLM strengths (context understanding, natural language interfaces) while mitigating weaknesses (spatial reasoning, physical constraint satisfaction) through validated computational frameworks. This realistic assessment positions CrystalEdit as a tool for accelerating certain workflows rather than replacing expert knowledge, while continuing to explore the boundaries of what LLM-assisted materials design can achieve.

\end{document}