% science_template.tex
% See accompanying readme.txt for copyright statement, change log etc.

% Any modification of this template, including writing a paper using it,
% MUST rename the file i.e. use a different file name.

%%%%%%%%%%%%%%%% START OF PREAMBLE %%%%%%%%%%%%%%%

% Basic setup. Authors shouldn't need to adjust these commands.
% It's annoying, but please do NOT strip these into a separate file.
% They need to be included in this .tex for our production software to work.

% Use the basic LaTeX article class, 12pt text
\documentclass[12pt]{article}

% Science uses Times font. If you don't have this installed (most LaTeX installations will be
% fine) or prefer the old Computer Modern fonts, comment out the following line
\usepackage{newtxtext,newtxmath}
% Depending on your LaTeX fonts installation, you might get better results with one or both of these:
% \usepackage{mathptmx}
%\usepackage{txfonts}

% Allow external graphics files
\usepackage{graphicx}

% Use US letter sized paper with 1 inch margins
\usepackage[letterpaper,margin=1in]{geometry}

% Yi Cao's additional 
\usepackage[version=4]{mhchem}
\usepackage{subcaption}
\usepackage{booktabs} % For better table formatting
% Double line spacing, including in captions
\linespread{1.5} % For some reason double spacing is 1.5, not 2.0!

% One space after each sentence
\frenchspacing

% Abstract formatting and spacing - no heading
\renewenvironment{abstract}
	{\quotation}
	{\endquotation}

% No date in the title section
% \date{}

% Reference section heading
\renewcommand\refname{References and Notes}

% Figure and Table labels in bold
\makeatletter
\renewcommand{\fnum@figure}{\textbf{Figure \thefigure}}
\renewcommand{\fnum@table}{\textbf{Table \thetable}}
\makeatother

% Call the accompanying scicite.sty package.
% This formats citation numbers in Science style.
\usepackage{scicite}

% Provides the \url command, and fixes a crash if URLs or DOIs contain underscores
\usepackage{url}

%%%%%%%%%%%% CUSTOM COMMANDS AND PACKAGES %%%%%%%%%%%%

% Authors can define simple custom commands e.g. as shortcuts to save on typing
% Use \newcommand (not \def) to avoid overwriting existing commands.
% Keep them as simple as possible and note the warning in the text below.
% Example:
\newcommand{\pcc}{\,cm$^{-3}$} % per cm-cubed

% Please DO NOT import additional external packages or .sty files.
% Those are unlikely to work with our conversion software and will cause problems later.
% Don't add any more \usepackage{} commands.


%%%%%%%%%%%%%%%% TITLE AND AUTHORS %%%%%%%%%%%%%%%%

% Title of the paper.
% Keep it short and understandable by any reader of Science.
% Avoid acronyms or jargon. Use sentence case.
\def\scititle{
	Project Update: CrystalEdit - Benchmarking LLMs for Physics-Aware Crystal Manipulation
}
% Store the title in a variable for reuse in the supplement (otherwise \maketitle deletes it)
\title{\bfseries \boldmath \scititle}

% Author and institution list.
% Institution numbers etc. should be hard-coded, do *not* use the \footnote command.
\author{
	% You can write out first names or use initials - either way is acceptable, but be consistent
	Yi Cao$^{1\dagger}$,
	Paulette Clancy$^{1\ast\dagger}$\and
	% Additional lines of authors should be inserted using the \and command (not \\
	% Institution list, in a slightly smaller font
	\small$^{1}$Chemical and Biomolecular Engineering, Johns Hopkins University, Baltimore & 21218, USA.\and
	% \small$^{2}$Another Department, Different Institution, City & Postal Code, Country.\and
	% % Identify at least one corresponding author, with contact email address
	\small$^{\ast}$Corresponding author. Email: pclancy3@jhu.edu\and
	% % Joint contributions can be indicated like this
	% \small$^{\dagger}$These authors contributed equally to this work.
}

%%%%%%%%%%%%%%%%% END OF PREAMBLE %%%%%%%%%%%%%%%%


%%%%%%%%%%%%%%%% START OF MAIN TEXT %%%%%%%%%%%%%%%
\begin{document} 

% Insert the title and author list
\maketitle

% The first paragraph of any Science paper does NOT have a heading
% Nor is it indented
\noindent


\section{Executive Summary: Why CrystalEdit?}
While generative AI has succeeded in \textit{de novo} crystal generation, the "Editing" problem remains unsolved. In the same way that a scientist doesn't always discover a new material but rather optimizes an existing one through doping or strain engineering, AI agents must learn to perform \textbf{targeted, physics-constrained modifications}.

\textbf{CrystalEdit} serves as a benchmark and a toolset to empower LLM agents to effectively move from \textit{understanding} how ML models see materials (your XAI work) to \textit{actively manipulating} those materials. It acts as the \textbf{"Active Intelligence"} component that completes our research ecosystem, bridging the gap between static datasets and the high-fidelity, non-equilibrium configurations required for robust Machine Learning Force Fields (MLFFs).

Key capabilities enabled by CrystalEdit:
\begin{itemize}
    \item \textbf{Bridge the "Data Gap":} Generate high-quality, non-equilibrium synthetic data for MLFF training without the prohibitive cost of long AIMD runs.
    \item \textbf{Automate Complex Workflows:} Reduce weeks of manual structure building (e.g., grain boundaries, twin defects) into natural language commands.
    \item \textbf{Enable Active Learning:} Directly link with our XAI findings to "edit" structures in directions that maximize the model's scientific information gain.
\end{itemize}

\section{Strategic Alignment with Current Research}
This project is the logical evolution of our previous work:

\begin{table}[h]
\centering
\begin{tabular}{@{}lp{10cm}@{}}
\toprule
\textbf{Previous Work} & \textbf{Connection to CrystalEdit} \\
\midrule
\textbf{NeurIPS (AI4Mat)} & We identified that MLFFs need diverse, non-equilibrium states. CrystalEdit provides the "surgical" precision to create these states (strains, defects) systematically. \\
\textbf{XAI (AAAI 2026)} & Our XAI framework tells us \textit{what} the model is learning; CrystalEdit allows us to \textit{test} those insights by specifically modifying the features the model relies on. \\
\textbf{General: Material Screening} & By automating the "Crystal Editing" process, we can move from screening known databases to screening "evolved" materials with nuanced modifications and defect manipulation. \\
\bottomrule
\end{tabular}
\caption{Strategic alignment of CrystalEdit with ongoing research goals.}
\label{tab:alignment}
\end{table}

\section{Implementation Progress}

\subsection{Stage 1: 0D Point Defects (The Building Blocks)}
We have automated the insertion of interstitials and substitutions in the \ce{Sb2Te3} system.
\begin{itemize}
    \item \textbf{Innovation:} The agent identifies octahedral/tetrahedral voids in the van der Waals gap automatically, ensuring steric and charge consistency.
    \item \textbf{Value:} Enables rapid "doping-on-the-fly" for digital twin experiments.
\end{itemize}

\subsection{Stage 2: 2D Surface \& Planar Engineering}
Critical for our work in topological insulators and 2D electronics.
\begin{itemize}
    \item \textbf{Slab Generation:} Controlled termination (Te vs. Sb) is now a single-command process (Figure \ref{fig:defects}A, B).
    \item \textbf{Stacking Faults:} We can now model metastable phases by inducing Shockley partial dislocations, enriching our MLFF training sets with transition-state-like configurations (Figure \ref{fig:defects}C).
\end{itemize}

\begin{figure}[ht]
    \centering
    \begin{subfigure}{0.8\textwidth}
        \includegraphics[width=\linewidth]{figures/Fig_Slab_Te.pdf}
        \caption{Te-Terminated Slab}
    \end{subfigure}
    \hfill
    \begin{subfigure}{0.8\textwidth}
        \includegraphics[width=\linewidth]{figures/Fig_Slab_Sb.pdf}
        \caption{Sb-Terminated Slab}
    \end{subfigure}
    \hfill
    \begin{subfigure}{0.8\textwidth}
        \includegraphics[width=\linewidth]{figures/Fig_Fault.pdf}
        \caption{Stacking Fault}
    \end{subfigure}
    \caption{Planar defects generated by CrystalEdit. (A) Thermodynamically stable Te-terminated surface. (B) Metastable Sb-terminated surface. (C) Stacking fault modifying the quintuple layer sequence.}
    \label{fig:defects}
\end{figure}

\subsection{Stage 3: Complex 3D Microstructures (The "Expert" Level)}
We have implemented Coincidence Site Lattice (CSL) theory for high-order defects.
\begin{itemize}
    \item \textbf{Twin Boundaries:} Automated construction of coherent $\Sigma 3$ and $\Sigma 7$ boundaries.
    \item \textbf{Hierarchical Logic:} The ability to nest defects (e.g., secondary \{332\} twins in BCC Ti) allows us to simulate the complex microstructures found in real-world industrial alloys (Figure \ref{fig:twin_render}).
\end{itemize}

\begin{figure}[ht]
    \centering
    \includegraphics[width=0.8\linewidth]{figures/twin_render_233_322.png}
    \caption{POV-Ray rendering of the hierarchical twin structure in BCC Ti. The image shows the primary \{233\} twin boundary plane (grey) and the distinct atomic arrangements of the twin domains (blue vs red), confirming the correct crystallographic orientation and periodicity.}
    \label{fig:twin_render}
\end{figure}

\section{Impact: Beyond Structure Building}
By benchmarking how well an LLM can perform these tasks, we are building a "Scientist-in-the-Loop" agent. This framework will allow us to:
\begin{enumerate}
    \item \textbf{Lower the Barrier:} Non-experts can generate Nature-quality simulation cells via natural language.
    \item \textbf{Ensure Physical Fidelity:} Every edit is passed through a "Physics Guardrail" (ASE/Pymatgen validation) before being accepted (Figure \ref{fig:benchmark}).
    \item \textbf{Synthesize Better Data:} Instead of "noisy" random perturbations, we generate "physics-guided" perturbations that represent real experimental conditions.
\end{enumerate}

\begin{figure}[ht]
    \centering
    \includegraphics[width=0.8\linewidth]{figures/Fig2_Benchmark.pdf}
    \caption{Benchmarking structural fidelity. The plot compares the 'Physics Score' (a composite metric of bond valence and steric validity) against the geometric complexity of the edit. CrystalEdit maintains high fidelity even for complex operations like grain boundary generation.}
    \label{fig:benchmark}
\end{figure}

\section{Conclusion}
This month's work has established CrystalEdit as a capable tool for deterministic crystal engineering, moving from simple point defects to complex, hierarchical microstructures. This "Active Intelligence" component is poised to accelerate our ability to explore and optimize material properties.

\end{document}